\documentclass[zihao=-4,UTF8]{ctexart}

\usepackage{amsmath}
\usepackage{graphicx}
\usepackage{float}
\usepackage{listing}
\usepackage{geometry}
\usepackage{multicol}

\geometry{left=3cm,right=2.5cm,top=2.5cm,bottom=2.5cm}

\linespread{1.8}



\begin{document}
\begin{figure}
    \begin{minipage}{0.6\linewidth}
        \includegraphics{pic/xiaohui.png}
    \end{minipage}
    \hfill
    \begin{minipage}{0.3\linewidth}
        组\ \ \ 别:\underline{ 本科生 }
        
        题\ \ \ 目:\underline{ B }

        队\ \ \ 号:\underline{ 030 }
    \end{minipage}
\end{figure}

\vbox{}
\vbox{}

\begin{figure}[H]
    \centering
    \includegraphics{pic/xiaoming.png}
\end{figure}
\vbox{}
\centerline{\textbf{\Huge{2019年数学建模竞赛}}}
\vbox{}
\vbox{}
\centerline{\LARGE{biaoti}}
\vbox{}
\vbox{}
\vbox{}

\large
\centerline{\underline{李亦龙 18373580}}\par
\centerline{\underline{叶凡 18374449}}\par
\centerline{\underline{栾帅 18373298}}\par
\centerline{队伍联系电话:\underline{13718250032}}\par
\centerline{队伍联系邮箱:\underline{18373580@buaa.edu.cn}}\par

\normalsize

\newpage
\section*{摘要}

\newpage

\section*{队伍声明}

我代表参赛队伍全体队员声明,本论文及其研究工作是由队伍成员独立完成的,在完成论文时所利用的一切资料均已在参考文献中列出 
\newpage
\tableofcontents

\section{问题重述}
基因共表达网络(Gene co-expression network)是现代生物基因工程研究的重要方向

\section{假设与符号}

\section{建立模型}

\section{求解问题}

\newpage
\section*{结论}


\newpage
\section*{附录}

\bibliography{library}
\section*{支撑材料文件列表}

\end{document}
